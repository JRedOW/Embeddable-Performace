\section{Methods}
This paper had two goals within its testing: to effectively test both embeddable and language performance and to be easily reproducible. By using multiple technologies, I believe I have achieved these goals.

\subsection{Reproducibility}
All resources required to run the benchmarks within these tests are contained with the git repository attached to this work or downloaded automatically when running the tests (excluding those required to download and build the testing environment). The testing environment is created and run in Docker containers, which are isolated, fresh, virtualized environments running at the computer's kernel level, reducing the impact of outside factors on performance. This also allows for them to be run on any computer or server with the proper software installed. To reproduce these tests, please read the README.md attached to the repository.

\subsection{Performance}
To limit external factors, all tests are written with minimal coding in individual files in an officially supported fashion of each embeddable language. Every test is compiled into an isolated executable. Static linking is used when possible, to improve performance. 

Each test file contains the required boilerplate code to initialize the embeddable language and give it the test script (code in an embeddable language) to run. If possible, it compiles the script into any forms for efficiency. It then runs the script 10,000 times, attempting to give a good picture of the performance of repeat processing of tasks.

\begin{listing}[H]
    \begin{minted}{C++}
    // HEADERS

    int main (int argc, const char * argv[])
    {
        lua_State* L = luaL_newstate();
        luaL_openlibs(L);
        luaL_dostring(L, 
            "function echo(x)\n"
            "  return x\n"
            "end\n"
        );

        std::string s = "bench";

        for (int i = 0; i < 100000; i++) {
            lua_getglobal(L, "echo");
            lua_pushlstring(L, s.c_str(), s.size());

            if (lua_pcall(L, 1, 1, 0) != 0) {
                luaL_error(L, "failed to call echo: %s", lua_tostring(L, -1));
            }

            std::string res(lua_tostring(L, -1));

            lua_pop(L, 1);
        }

        return 0;
    }
    \end{minted}
    \caption{The Echo Test For LuaJIT}
    \label{lst:luajit-echo}
\end{listing}

All tests are run by the software hyperfine\cite{hyperfine} which tracks how long an executable takes to run. The executable is first run 100 times for "warmup," as greater fluctuations were observed without these runs. The executable is then run 500 times while being measured, giving a good picture of the mean and standard deviation of the execution time. These results are then written to a json file that can be parsed for greater study and visualizations.

\subsubsection{Embeddable Performance}
To test embeddable performance, their will be one test. The embeddable language must take a string (all tests use the string "bench") as a parameter of a function, and simply return it. This tests three separate subsets of embeddable language performance on top of the systems used to initiate itself: the speed at which the the embeddable language can take data from the host language, the speed at which the embeddable language can send data to the host language, and the performance of converting strings, if necessary, from/to the host language. An example of this was shown in Listing~\ref{lst:luajit-echo}.
